\documentclass{article}

\usepackage{geometry}
\usepackage{graphicx}
\usepackage{amssymb}
\usepackage{amsmath}
\usepackage{amsthm}
\usepackage{empheq}
\usepackage{mdframed}
\usepackage{booktabs}
\usepackage{color}
\usepackage{psfrag}
% \usepackage{arabtex}
% \usepackage{utf8}
% \setcode{utf8}

% Other Settings

\geometry{a4paper}

\definecolor{ocre}{RGB}{243,102,25}
\definecolor{mygray}{RGB}{243,243,244}
\definecolor{deepGreen}{RGB}{26,111,0}
\definecolor{shallowGreen}{RGB}{235,255,255}
\definecolor{deepBlue}{RGB}{61,124,222}
\definecolor{shallowBlue}{RGB}{235,249,255}

\newcommand\orangebox[1]{\fcolorbox{ocre}{mygray}{\hspace{1em}#1\hspace{1em}}}

\newtheoremstyle{mytheoremstyle}{3pt}{3pt}{\normalfont}{0cm}{\rmfamily\bfseries}{}{1em}{\color{black}\thmname{#1}}
\newtheoremstyle{myexamplestyle}{3pt}{3pt}{\normalfont}{0cm}{\rmfamily\bfseries}{}{1em}{\color{black}\thmname{#1}}
\theoremstyle{mytheoremstyle}
\newmdtheoremenv[linewidth=1pt,backgroundcolor=shallowGreen,linecolor=deepGreen,leftmargin=0pt,innerleftmargin=20pt,innerrightmargin=20pt,]{theorem}{Theorem}[section]
\theoremstyle{mytheoremstyle}
\newmdtheoremenv[linewidth=1pt,backgroundcolor=shallowBlue,linecolor=deepBlue,leftmargin=0pt,innerleftmargin=20pt,innerrightmargin=20pt,]{definition}{Definition}
\theoremstyle{myexamplestyle}
\newmdtheoremenv[linecolor=black,leftmargin=0pt,innerleftmargin=10pt,innerrightmargin=10pt,]{example}{Example}

% \title{\<مدرسه رياضيات>}
\title{Logic}


\begin{document}
\maketitle

\tableofcontents

\newpage

\section{Introduction}

\subsection{Truth value}

\begin{definition}
    A truth value is a value that indicates the truth or falsity of a statement or proposition. It is usually represented as "true" or "false."
\end{definition}

\begin{example}
    \emph{The sky is blue}. When you look up during the day, the sky is indeed blue. So, the truth value of this statement is \emph{true}. It's like answering a question with \emph{yes} or \emph{no}.
\end{example}

\begin{example}
    \emph{Cats can fly}. The truth value of this statement is \emph{false} because cats cannot fly. So, a truth value helps us know whether a statement or idea is correct (true) or incorrect (false).
\end{example}


\subsection{Statement}

\begin{definition}
    A statement or proposition is a sentence that says something
    that can be either true or false, but not both. It's like
    claiming something.
\end{definition}

\begin{example}
    \emph{The sun rises in the east}. This is a statement that can be checked to see if it's true or false.
\end{example}

\begin{example}
    \emph{Elephants are smaller than mice}. This is another statement, and we can check if it's true or false.
\end{example}

In other words, a statement is any sentence that tells us something about the world and has a truth value, meaning it can be judged as true or false.


\newpage

\subsection{Logic}

\begin{definition}
    Logic is a way of thinking that helps us determine if statements or propositions are true or false. It uses rules and principles to analyze arguments and reasoning.
\end{definition}

Here's a simple way to think about it:

\begin{enumerate}
    \item \textbf{Reasoning}: Logic helps us think clearly and make good decisions. For example, if you know that "All birds have feathers" and "A robin is a bird," logic helps you conclude that "A robin has feathers."

    \item \textbf{Rules}: Logic follows specific rules to make sure our thinking is correct. These rules help us avoid mistakes in our reasoning.

    \item \textbf{Truth}: Logic helps us understand what is true and what is false. It provides a structured way to figure out the truth value of statements or propositions.
\end{enumerate}


\begin{example}
    If we know that \emph{If it rains, the ground gets wet}, and we
    observe that \emph{It is raining}, we can logically conclude
    that \emph{The ground will get wet}.
\end{example}

Logic is used in many areas, like math, computer science, and everyday problem-solving, to ensure our conclusions are based on solid reasoning.

\section{Connectives}

\begin{definition}
    Connectives, also called logical operators, are words or symbols
    that connect statements to form more complex statements.
\end{definition}
They help us understand the relationships between different propositions and determine the overall truth value.


\subsection{Negation}

\begin{definition}
    Negates a statement, making it the opposite of its original truth value.
\end{definition}

\begin{example}
    \emph{It is NOT raining.} If \emph{It is raining} is true, then \emph{It is NOT raining} is false, and vice versa.
\end{example}

\subsection{Conjunction}
\begin{definition}
    Combines two statements and is true only if both statements are
    true.
\end{definition}

\begin{example}
    \emph{It is raining AND it is cold.} This is true only if both \emph{It is raining} is true and \emph{It is cold} is true.
\end{example}

\newpage

\subsection{Disjunction}

\begin{definition}
    Combines two statements and is true if at least one of the
    statements is true.
\end{definition}

\begin{example}
    \emph{It is raining OR it is sunny.} This is true if either
    \emph{It is raining} is true or \emph{It is sunny} is true, or
    if both are true.
\end{example}

\subsection{Implication}

\begin{definition}
    Indicates that if the first statement is true, then the second
    statement must also be true.
\end{definition}

\begin{example}
    \emph{IF it rains, THEN the ground gets wet}. This means if
    \emph{It rains} is true, then \emph{The ground gets wet} must
    also be true.
\end{example}

There are different ways of rephrasing an implication. Let's look at
each of these with examples to make them clear. Keep in mind an implication has the form \emph{If P, then Q}.

\subsubsection{Contrapositive}
\begin{definition}
    The contrapositive of the implication \emph{If P, then Q} is
    \emph{If not Q, then not P}. The contrapositive is always
    logically equivalent to the original statement, meaning they are
    either both true or false.
\end{definition}

\begin{example}
    \emph{If the ground is not wet (not Q), then it does not rain (not P)}.
\end{example}

\subsubsection{Converse}

\begin{definition}
    The converse of the implication \emph{If P, then Q} is \emph{If
        Q, then P}. The converse is not necessarily logically equivalent
    to the original statement.
\end{definition}

\begin{example}
    \emph{If the ground gets wet (Q), then it rains (P)}. This might
    not always be true because the ground could get wet for other
    reasons, like someone watering the garden.
\end{example}

\newpage

\subsubsection{Inverse}

\begin{definition}
    The inverse of the implication \emph{If P, then Q} is \emph{If not P, then not Q}. The inverse is also not necessarily logically equivalent to the original statement.
\end{definition}

\begin{example}
    \emph{If it does not rain (not P), then the ground does not get wet (not Q)}. This might not always be true for the same reason as the converse—the ground could get wet in other ways.
\end{example}


\begin{table}[h!]
    \centering
    \begin{tabular}{|l|l|}
        \hline
        \textbf{Statement} & \textbf{Example}      \\\hline
        Original Statement & If P, then Q.         \\\hline
        Contrapositive     & If not Q, then not P. \\\hline
        Converse           & If Q, then P.         \\\hline
        Inverse            & If not P, then not Q. \\\hline
    \end{tabular}
    \caption{Different ways of rephrasing an implication}
\end{table}

\subsection{Tautology}
\begin{definition}
    A tautology is a statement that is always true, regardless of the
    truth values of its components.
\end{definition}

\begin{example}
    \emph{All birds have feathers}.
\end{example}

\subsection{Contradiction}

\begin{definition}
    A contradiction is a statement that is always false, regardless of the truth values of its components.
\end{definition}

\begin{example}
    \emph{The sky is blue AND the sky is not blue} is a
    contradiction because the sky can't be both blue and not blue at
    the same time.
\end{example}

\newpage

\section{Quantifiers}

\begin{definition}
    Quantifiers are symbols or words in logic that express the
    quantity of specimens in the domain of discourse that satisfy a
    certain property.
\end{definition}

They help us talk about \emph{how many} things satisfy a given
condition. There are two main types of quantifiers:

\subsection{Universal Quantifier}

\begin{definition}
    The universal quantifier, denoted by the symbol $\forall$,
    expresses that a statement is true for all elements in a set or
    domain. It means \emph{for all} or \emph{for every}.
\end{definition}


\begin{example}
    $\forall x \in \mathbb{N}, x > 0$ reads as \emph{for all $x$ in
        the set of natural numbers, $x$ is greater than 0}. This
    statement is true because all natural numbers are greater than 0.
\end{example}

\subsection{Existential Quantifier}
\begin{definition}
    The existential quantifier, denoted by the symbol $\exists$,
    expresses that a statement is true for at least one element in a
    set or domain. It means \emph{there exists} or \emph{there is}.
\end{definition}


\begin{example}
    $\exists x \in \mathbb{N}, x > 0$ reads as \emph{There
        exists an $x$ in the set of natural numbers such that $x$ is greater
        than 0}. This statement is true because there are natural
    numbers greater than 0.
\end{example}

\subsection{Negation of Universal Quantifier}

The negation of a universal quantifier is an existential quantifier
with a negated statement. Consider the statement \emph{All crows are
    black}. The negation of this statement is \emph{Not all crows
    are black}, which can be rewritten as \emph{There exists a crow that is not black}.

\subsection{Negation of Existential Quantifier}

The negation of an existential quantifier is a universal quantifier
with a negated statement. Consider the statement \emph{Some crows
    are black}. The negation of this statement is \emph{No crows are
    black}, which can be rewritten as \emph{For all crows, they are not black}.

\newpage

\section{Proofs}

Proofs are logical arguments that demonstrate the truth of a
statement or proposition. They are used in mathematics, logic, and
other fields to establish that something is true based on previously
known facts, axioms, and logical reasoning. Here's a breakdown of
what proofs are and how they work:

Key Components of a Proof:
\begin{itemize}
    \item \textbf{Axioms or Premises}: These are the starting points or assumptions that are accepted as true without proof. In mathematics, axioms are basic truths about numbers and operations.
    \item \textbf{Theorems or Propositions}: These are statements or claims that we want to prove. A theorem is often a significant result, while a proposition is usually a less important but still valid result.
    \item \textbf{Logical Reasoning}: This involves using rules of logic to connect axioms and previously proven theorems to the statement we want to prove.
    \item \textbf{Conclusion}: The end result of the proof, showing that the theorem or proposition is true.
\end{itemize}

\subsection{Direct Proof}

This type of proof involves directly showing that a statement is
true by using logical steps from the premises to the conclusion.

\begin{example}
    Prove that if \( n \) is an even number, then \( n^2 \) is even.
    \begin{proof}
        Assume \( n \) is even. Then \( n = 2k \) for some integer \( k \). So, \( n^2 = (2k)^2 = 4k^2 = 2(2k^2) \). Since \( 2k^2 \) is an integer, \( n^2 \) is even.
    \end{proof}
\end{example}


\subsection{Proof by Cases}

Proof by cases involves breaking a proof into several distinct cases and proving that the statement holds in each case.


\begin{example}
    Prove that for any integer \( n \), \( n^2 \) is either a multiple of 4 or one more than a multiple of 4.

    \begin{proof}
        Consider the integer \( n \). We need to consider two cases based on the parity (whether \( n \) is even or odd) of \( n \).

        \textbf{Case 1: \( n \) is even}.
        If \( n \) is even, we can write \( n = 2k \) for some
        integer \( k \), then \( n^2 = (2k)^2 = 4k^2 \). Here,
        \( n^2 \) is a multiple of 4.

        \textbf{Case 2: \( n \) is odd}.
        If \( n \) is odd, we can write \( n = 2k + 1 \) for some integer \( k \). Then
        \( n^2 = (2k + 1)^2 = 4k^2 + 4k + 1 = 4(k^2 + k) + 1 \).
        Here, \( n^2 \) is one more than a multiple of 4.

        Since in both cases \( n^2 \) is either a multiple of 4 or one more than a multiple of 4, the statement is proved.
    \end{proof}
\end{example}

\newpage

\subsection{Proof by Contrapositive}

Proof by contrapositive involves proving the contrapositive of the statement. The contrapositive of \emph{If \( P \), then \( Q \)} is \emph{If not \( Q \), then not \( P \)}. The contrapositive is logically equivalent to the original statement.


\begin{example}
    Prove that if \( n^2 \) is even, then \( n \) is even.

    \begin{proof}

        We will prove the contrapositive:

        \begin{center}
            \emph{If \( n \) is not even, then \( n^2 \) is not even.}
        \end{center}

        Assume \( n \) is odd, then \( n = 2k + 1 \) for some integer \( k \).

        \[ n^2 = (2k + 1)^2 = 4k^2 + 4k + 1 = 2(2k^2 + 2k) + 1 \]

        Here, \( n^2 \) is of the form \( 2m + 1 \) for some integer \( m \), which means \( n^2 \) is odd.

        Since we have shown that \emph{If \( n \) is odd, then \( n^2 \) is odd}, we have proven the contrapositive. Thus, the original statement \emph{If \( n^2 \) is even, then \( n \) is even} is also true.
    \end{proof}
\end{example}

\subsection{Proof by Contradiction}

This type of proof involves assuming the opposite of what you want to prove and showing that this assumption leads to a contradiction.


\begin{example}
    Prove that there is no largest prime number.
    \begin{proof}
        Assume there is a largest prime number, \( p \). Consider
        the number \( N = p! + 1 \). \( N \) is not divisible by any
        prime number less than or equal to \( p \), which means
        \( N \) is either prime itself or divisible by a prime
        larger than \( p \). This contradicts the assumption that
        \( p \) is the largest prime.
    \end{proof}
\end{example}

\subsection{Proof by Induction}

This type of proof is used to prove statements about all natural numbers. It involves two steps: the base case and the inductive step.

\begin{example}
    Prove that the sum of the first \( n \) natural numbers is
    \( \displaystyle\frac{n(n+1)}{2} \).

    \begin{proof}

        \textbf{Base case:} For \( n = 1 \), the sum is
        \(\displaystyle 1 = \frac{1(1+1)}{2} \), which is true.

        \textbf{Inductive step:} Assume the formula is true for some \( n = k \). Then the sum of the first \( k \) numbers is \( \frac{k(k+1)}{2} \). For \( n = k+1 \), the sum is

        \[\frac{k(k+1)}{2} + (k+1) = \frac{k(k+1) + 2(k+1)}{2} = \frac{(k+1)(k+2)}{2}\]

        Thus, the formula holds for \( k+1 \).
    \end{proof}
\end{example}

\end{document}

